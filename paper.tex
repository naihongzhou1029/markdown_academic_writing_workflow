% Options for packages loaded elsewhere
\PassOptionsToPackage{unicode}{hyperref}
\PassOptionsToPackage{hyphens}{url}
\PassOptionsToPackage{space}{xeCJK}
\documentclass[
]{article}
\usepackage{xcolor}
\usepackage{amsmath,amssymb}
\setcounter{secnumdepth}{5}
\usepackage{iftex}
\ifPDFTeX
  \usepackage[T1]{fontenc}
  \usepackage[utf8]{inputenc}
  \usepackage{textcomp} % provide euro and other symbols
\else % if luatex or xetex
  \usepackage{unicode-math} % this also loads fontspec
  \defaultfontfeatures{Scale=MatchLowercase}
  \defaultfontfeatures[\rmfamily]{Ligatures=TeX,Scale=1}
\fi
\usepackage{lmodern}
\ifPDFTeX\else
  % xetex/luatex font selection
  \ifXeTeX
    \usepackage{xeCJK}
    \setCJKmainfont[]{Noto Sans CJK SC}
  \fi
  \ifLuaTeX
    \usepackage[]{luatexja-fontspec}
    \setmainjfont[]{Noto Sans CJK SC}
  \fi
\fi
% Use upquote if available, for straight quotes in verbatim environments
\IfFileExists{upquote.sty}{\usepackage{upquote}}{}
\IfFileExists{microtype.sty}{% use microtype if available
  \usepackage[]{microtype}
  \UseMicrotypeSet[protrusion]{basicmath} % disable protrusion for tt fonts
}{}
\makeatletter
\@ifundefined{KOMAClassName}{% if non-KOMA class
  \IfFileExists{parskip.sty}{%
    \usepackage{parskip}
  }{% else
    \setlength{\parindent}{0pt}
    \setlength{\parskip}{6pt plus 2pt minus 1pt}}
}{% if KOMA class
  \KOMAoptions{parskip=half}}
\makeatother
\usepackage{longtable,booktabs,array}
\usepackage{calc} % for calculating minipage widths
% Correct order of tables after \paragraph or \subparagraph
\usepackage{etoolbox}
\makeatletter
\patchcmd\longtable{\par}{\if@noskipsec\mbox{}\fi\par}{}{}
\makeatother
% Allow footnotes in longtable head/foot
\IfFileExists{footnotehyper.sty}{\usepackage{footnotehyper}}{\usepackage{footnote}}
\makesavenoteenv{longtable}
\usepackage{graphicx}
\makeatletter
\newsavebox\pandoc@box
\newcommand*\pandocbounded[1]{% scales image to fit in text height/width
  \sbox\pandoc@box{#1}%
  \Gscale@div\@tempa{\textheight}{\dimexpr\ht\pandoc@box+\dp\pandoc@box\relax}%
  \Gscale@div\@tempb{\linewidth}{\wd\pandoc@box}%
  \ifdim\@tempb\p@<\@tempa\p@\let\@tempa\@tempb\fi% select the smaller of both
  \ifdim\@tempa\p@<\p@\scalebox{\@tempa}{\usebox\pandoc@box}%
  \else\usebox{\pandoc@box}%
  \fi%
}
% Set default figure placement to htbp
\def\fps@figure{htbp}
\makeatother
% definitions for citeproc citations
\NewDocumentCommand\citeproctext{}{}
\NewDocumentCommand\citeproc{mm}{%
  \begingroup\def\citeproctext{#2}\cite{#1}\endgroup}
\makeatletter
 % allow citations to break across lines
 \let\@cite@ofmt\@firstofone
 % avoid brackets around text for \cite:
 \def\@biblabel#1{}
 \def\@cite#1#2{{#1\if@tempswa , #2\fi}}
\makeatother
\newlength{\cslhangindent}
\setlength{\cslhangindent}{1.5em}
\newlength{\csllabelwidth}
\setlength{\csllabelwidth}{3em}
\newenvironment{CSLReferences}[2] % #1 hanging-indent, #2 entry-spacing
 {\begin{list}{}{%
  \setlength{\itemindent}{0pt}
  \setlength{\leftmargin}{0pt}
  \setlength{\parsep}{0pt}
  % turn on hanging indent if param 1 is 1
  \ifodd #1
   \setlength{\leftmargin}{\cslhangindent}
   \setlength{\itemindent}{-1\cslhangindent}
  \fi
  % set entry spacing
  \setlength{\itemsep}{#2\baselineskip}}}
 {\end{list}}
\usepackage{calc}
\newcommand{\CSLBlock}[1]{\hfill\break\parbox[t]{\linewidth}{\strut\ignorespaces#1\strut}}
\newcommand{\CSLLeftMargin}[1]{\parbox[t]{\csllabelwidth}{\strut#1\strut}}
\newcommand{\CSLRightInline}[1]{\parbox[t]{\linewidth - \csllabelwidth}{\strut#1\strut}}
\newcommand{\CSLIndent}[1]{\hspace{\cslhangindent}#1}
\setlength{\emergencystretch}{3em} % prevent overfull lines
\providecommand{\tightlist}{%
  \setlength{\itemsep}{0pt}\setlength{\parskip}{0pt}}
\pagenumbering{arabic}
\setcounter{page}{1}
\usepackage{xeCJK}
\setCJKmainfont{Noto Sans CJK SC}
\usepackage[a4paper,margin=1in]{geometry}
\usepackage{etoolbox}
\AtBeginEnvironment{CSLReferences}{%
  \newpage\section*{References}%
  \setlength{\parindent}{0pt}%
}
\pretocmd{\tableofcontents}{\clearpage}{}{}
\pretocmd{\listoffigures}{\clearpage}{}{}
\pretocmd{\listoftables}{\clearpage}{}{}
\apptocmd{\listoftables}{\clearpage}{}{}
\makeatletter
\@ifpackageloaded{subfig}{}{\usepackage{subfig}}
\@ifpackageloaded{caption}{}{\usepackage{caption}}
\captionsetup[subfloat]{margin=0.5em}
\AtBeginDocument{%
\renewcommand*\figurename{Figure}
\renewcommand*\tablename{Table}
}
\AtBeginDocument{%
\renewcommand*\listfigurename{List of Figures}
\renewcommand*\listtablename{List of Tables}
}
\newcounter{pandoccrossref@subfigures@footnote@counter}
\newenvironment{pandoccrossrefsubfigures}{%
\setcounter{pandoccrossref@subfigures@footnote@counter}{0}
\begin{figure}\centering%
\gdef\global@pandoccrossref@subfigures@footnotes{}%
\DeclareRobustCommand{\footnote}[1]{\footnotemark%
\stepcounter{pandoccrossref@subfigures@footnote@counter}%
\ifx\global@pandoccrossref@subfigures@footnotes\empty%
\gdef\global@pandoccrossref@subfigures@footnotes{{##1}}%
\else%
\g@addto@macro\global@pandoccrossref@subfigures@footnotes{, {##1}}%
\fi}}%
{\end{figure}%
\addtocounter{footnote}{-\value{pandoccrossref@subfigures@footnote@counter}}
\@for\f:=\global@pandoccrossref@subfigures@footnotes\do{\stepcounter{footnote}\footnotetext{\f}}%
\gdef\global@pandoccrossref@subfigures@footnotes{}}
\@ifpackageloaded{float}{}{\usepackage{float}}
\floatstyle{ruled}
\@ifundefined{c@chapter}{\newfloat{codelisting}{h}{lop}}{\newfloat{codelisting}{h}{lop}[chapter]}
\floatname{codelisting}{Listing}
\newcommand*\listoflistings{\listof{codelisting}{List of Listings}}
\makeatother
\usepackage{bookmark}
\IfFileExists{xurl.sty}{\usepackage{xurl}}{} % add URL line breaks if available
\urlstyle{same}
\hypersetup{
  pdftitle={Sustainable Scholarship: A Robust Academic Workflow with Markdown, Pandoc, and LaTeX},
  pdfauthor={An Old-Fashioned Researcher},
  hidelinks,
  pdfcreator={LaTeX via pandoc}}

\title{Sustainable Scholarship: A Robust Academic Workflow with
Markdown, Pandoc, and LaTeX}
\author{An Old-Fashioned Researcher}
\date{2025-12-06}

\begin{document}
\maketitle
\begin{abstract}
This document serves as a complete, self-referential example of the
Pandoc academic workflow. It demonstrates the use of a YAML metadata
block for configuration, automated citations with pandoc-citeproc,
cross-references for figures, tables, and equations with
pandoc-crossref, and advanced customization for multilingual typesetting
and page numbering through LaTeX. Furthermore, it details recent
extensions to the system, including a Docker-based reproducible build
environment, programmatic diagramming with Mermaid, and an automated
LLM-powered translation pipeline for multilingual publishing.
\end{abstract}

{
\setcounter{tocdepth}{2}
\tableofcontents
}
\listoffigures
\listoftables
\section{Introduction: The Philosophy of Plain-Text
Academia}\label{introduction-the-philosophy-of-plain-text-academia}

The choice of writing tools in academic work is not merely a matter of
technical preference; it is a philosophical commitment to a particular
mode of scholarship. The modern plain-text workflow, centered on
Markdown, Pandoc, and LaTeX, represents a deliberate move towards a more
robust, transparent, and durable scholarly practice
(\citeproc{ref-healy2018plain}{Healy 2018}). This approach is founded on
a set of core principles that stand in stark contrast to the opaque,
proprietary nature of traditional word processors.

At its heart is the principle of \textbf{sustainability}. Plain-text
files, such as those written in Markdown, are the most resilient digital
format. Unlike the complex binary structures of \texttt{.docx} files,
which can become corrupted or obsolete as software changes, plain text
will remain readable and accessible for decades, ensuring the long-term
viability of one's intellectual output
(\citeproc{ref-healy2018plain}{Healy 2018};
\citeproc{ref-macfarlane2022pandoc}{MacFarlane 2022}). This is coupled
with the powerful concept of \textbf{separation of concerns}, where the
semantic content of a document---the text, its structure, and its
meaning---is written in Markdown, completely divorced from its final
visual presentation, which is handled independently by LaTeX templates
or other styling mechanisms. This modularity allows for radical changes
in output format with zero alteration to the source manuscript.

Furthermore, this workflow is uniquely suited for the rigorous demands
of modern research. Plain-text files integrate seamlessly with
\textbf{version control systems} like Git, enabling meticulous tracking
of every change, non-destructive experimentation with drafts, and
transparent collaboration among authors---a process notoriously fraught
with difficulty when using binary files
(\citeproc{ref-healy2018plain}{Healy 2018}). The entire process, from
the initial draft to the final PDF, can be automated with simple
scripts, ensuring perfect \textbf{reproducibility} at any point in the
future, a cornerstone of scientific and scholarly integrity.

This system is best understood not as a collection of disparate tools,
but as a linear, modular data processing pipeline. The raw manuscript
(\texttt{.md}) and bibliographic data (\texttt{.json} or \texttt{.bib})
serve as the initial inputs. These inputs are then passed through a
chain of specialized filters and transformers: Zotero and its Better
BibTeX extension manage and export bibliographic data; Pandoc parses the
source text (\citeproc{ref-macfarlane2022pandoc}{MacFarlane 2022});
\texttt{pandoc-citeproc} resolves citation markers;
\texttt{pandoc-crossref} numbers figures and equations
(\citeproc{ref-lierdakil2021crossref}{Lierdakil 2021}); and finally, a
LaTeX engine like XeLaTeX performs the final typesetting to produce a
PDF.

\begin{figure}
\centering
\pandocbounded{\includegraphics[keepaspectratio,alt={Mermaid diagram}]{images/mermaid-1.png}}
\caption{Mermaid diagram}
\end{figure}

Each stage is discrete and transparent. This contrasts fundamentally
with the monolithic, ``black box'' environment of a word processor,
where these processes are intertwined and hidden from the user. The
power of this workflow lies in the ability to control this pipeline, to
swap out components, insert new processing stages, and debug any issue
by inspecting the intermediate output at any point---for instance, by
generating the intermediate \texttt{.tex} file to diagnose a LaTeX
error. This level of control and transparency is the key to solving the
complex, bespoke formatting challenges inherent in academic writing.

\section{Part I: The Core Toolchain}\label{part-i-the-core-toolchain}

\subsection{Section 1: Assembling Your Digital
Workbench}\label{section-1-assembling-your-digital-workbench}

Before embarking on the plain-text writing process, a robust environment
is required. While one can install tools individually, this project
demonstrates a modern, \textbf{containerized approach} to academic
writing.

\textbf{Docker: The Reproducible Environment} A significant challenge in
academic workflows is dependency management---ensuring that Pandoc,
LaTeX packages, and helper scripts work consistently across different
machines. To solve this, we utilize \textbf{Docker}. The entire
toolchain is encapsulated in a Docker image (derived from
\texttt{dalibo/pandocker}). This ensures that the build process is
perfectly reproducible: if it builds on one machine, it will build on
any machine with Docker installed. The complex ``plumbing'' of the
workflow is abstracted away behind simple wrapper scripts
(\texttt{./make-docker.sh}), allowing the author to focus solely on
writing.

\textbf{The Scripted Build System} Behind the scenes, the
\texttt{Makefile} orchestrates the build process. To manage complexity
and ensure cross-platform compatibility (Linux, macOS, Windows),
implementation details have been extracted into modular shell scripts in
the \texttt{tools/} directory. These scripts handle tasks like font
detection, PDF merging, and dependency checks, running transparently
inside the Docker container.

\textbf{A Plain-Text Editor: The Writing Environment} The writing itself
is done in a plain-text editor. Modern, extensible editors like Visual
Studio Code (VSCode), Atom, or the academic-focused Zettlr are ideal
choices.

\textbf{Zotero: The Reference Manager} Robust reference management is
handled by Zotero, a powerful, open-source application. For this
workflow, the installation of the \textbf{Better BibTeX for Zotero
(BBT)} extension is non-negotiable. BBT provides two essential features:
the automatic generation of stable, human-readable citation keys and a
highly reliable, automated export process that keeps the bibliography
file synchronized with the Zotero library.

The components are summarized in Tab.~\ref{tbl:workbench}.

\begin{longtable}[]{@{}
  >{\raggedright\arraybackslash}p{(\linewidth - 6\tabcolsep) * \real{0.1006}}
  >{\raggedright\arraybackslash}p{(\linewidth - 6\tabcolsep) * \real{0.3128}}
  >{\raggedright\arraybackslash}p{(\linewidth - 6\tabcolsep) * \real{0.1117}}
  >{\raggedright\arraybackslash}p{(\linewidth - 6\tabcolsep) * \real{0.4749}}@{}}
\caption{\label{tbl:workbench}The Digital Scholar's
Workbench.}\tabularnewline
\toprule\noalign{}
\begin{minipage}[b]{\linewidth}\raggedright
Component
\end{minipage} & \begin{minipage}[b]{\linewidth}\raggedright
Purpose
\end{minipage} & \begin{minipage}[b]{\linewidth}\raggedright
Recommended Software
\end{minipage} & \begin{minipage}[b]{\linewidth}\raggedright
Key Configuration Notes
\end{minipage} \\
\midrule\noalign{}
\endfirsthead
\toprule\noalign{}
\begin{minipage}[b]{\linewidth}\raggedright
Component
\end{minipage} & \begin{minipage}[b]{\linewidth}\raggedright
Purpose
\end{minipage} & \begin{minipage}[b]{\linewidth}\raggedright
Recommended Software
\end{minipage} & \begin{minipage}[b]{\linewidth}\raggedright
Key Configuration Notes
\end{minipage} \\
\midrule\noalign{}
\endhead
\bottomrule\noalign{}
\endlastfoot
Container Runtime & Provides a consistent, reproducible build
environment. & Docker & Use \texttt{./make-docker.sh} to run builds
without local dependency issues. \\
Document Converter & Parses Markdown and orchestrates the conversion
process. & Pandoc (in Docker) & No local installation required when
using the Docker workflow. \\
Typesetting Engine & Compiles LaTeX code generated by Pandoc into a PDF.
& TeX Live (in Docker) & Included in the Docker image. \\
Text Editor & The environment for writing in plain-text Markdown. &
VSCode & Install extensions: Markdown Preview Enhanced and Pandoc
Citer. \\
Reference Manager & Manages bibliographic data and exports it for
Pandoc. & Zotero & Install the Better BibTeX for Zotero (BBT) extension
for stable keys and auto-export. \\
\end{longtable}

\subsection{Section 2: The Pandoc Conversion Engine: From Markdown to
PDF}\label{section-2-the-pandoc-conversion-engine-from-markdown-to-pdf}

With the toolchain installed, the conversion process is driven by the
Pandoc command-line interface, configured primarily through a metadata
block within the Markdown file itself, as seen at the top of this very
document.

\textbf{The Basic Conversion Command} The fundamental command to convert
a Markdown file to a PDF is straightforward:
\texttt{pandoc\ input.md\ -o\ output.pdf}. Pandoc typically infers the
input and output formats from the file extensions.

\textbf{The YAML Metadata Block: The Document's Control Panel} Rather
than relying on long and cumbersome command-line flags, Pandoc
configurations are best managed within a YAML metadata block at the very
top of the Markdown file, delimited by \texttt{-\/-\/-} on either side.
This approach is superior because it keeps the document's essential
metadata and its compilation settings version-controlled alongside the
content itself.

\textbf{The Standalone Flag (\texttt{-s})} A critical option for
generating a complete document is \texttt{-\/-standalone} (or its
shorthand, \texttt{-s}). This flag instructs Pandoc to use a template to
wrap the converted content with the necessary header and footer
material---for example, the
\texttt{\textbackslash{}documentclass\{...\}} and
\texttt{\textbackslash{}begin\{document\}...\textbackslash{}end\{document\}}
commands in LaTeX---to create a self-contained, compilable file rather
than a mere fragment.

\section{Part II: Managing Scholarly
Apparatus}\label{part-ii-managing-scholarly-apparatus}

\subsection{\texorpdfstring{Section 3: Automated Citations and
Bibliographies with
\texttt{pandoc-citeproc}}{Section 3: Automated Citations and Bibliographies with pandoc-citeproc}}\label{section-3-automated-citations-and-bibliographies-with-pandoc-citeproc}

A cornerstone of academic writing is the correct management of citations
and bibliographies. The Pandoc workflow automates this process with
exceptional precision using its citation processor,
\texttt{pandoc-citeproc}.

\textbf{The Role of \texttt{pandoc-citeproc}} The citation processor,
invoked with the \texttt{-\/-citeproc} flag in modern Pandoc versions,
is the filter responsible for parsing Markdown citation syntax,
transforming it into fully formatted in-text citations, and
automatically generating a bibliography at the end of the document.

\textbf{Step 1: Configuring Zotero and Better BibTeX (BBT)} The process
begins with the reference manager. Within Zotero, the Better BibTeX
extension should be configured to automatically export the desired
library or collection to a file whenever an entry is added or modified.

\textbf{Step 2: Choosing the Bibliography Format: CSL JSON vs.~BibTeX}
To ensure the highest possible fidelity between the reference manager
and the final document, it is strongly recommended to export directly
from Zotero to a CSL-native format, such as \textbf{Better CSL JSON} or
\textbf{Better CSL YAML}. This eliminates the ``man-in-the-middle''
conversion and preserves the integrity of the bibliographic data.

\textbf{Step 3: Specifying Bibliography and Style in YAML} The
connection between the Markdown document and the bibliographic data is
made in the YAML metadata block. Two keys are required:
\texttt{bibliography:} and \texttt{csl:}. Thousands of CSL styles for
various journals and conventions are available for download from the
official Zotero Style Repository.

\textbf{Step 4: Mastering Pandoc Citation Syntax} With the setup
complete, citing sources in Markdown is simple and expressive. The
syntax supports a wide range of scholarly conventions.

\begin{itemize}
\item
  \textbf{Standard Parenthetical Citations:}
  \texttt{...as\ has\ been\ shown\ {[}@knuth1984tex;\ @lamport1986latex{]}}.
\item
  \textbf{Narrative (Author-in-Text) Citations:}
  \texttt{@macfarlane2022pandoc\ argues\ that...} renders as
  ``MacFarlane (2022) argues that\ldots{}''.
\item
  \textbf{Suppressing the Author:} When an author is already mentioned,
  their name can be suppressed in the citation by adding a minus sign:
  \texttt{Healy\textquotesingle{}s\ research\ {[}-@healy2018plain{]}\ confirms...}
  renders as ``Healy's research (2018) confirms\ldots{}''.
\item
  \textbf{Prefixes, Locators, and Suffixes:}
  \texttt{{[}see\ @knuth1984tex,\ chap.\ 3{]}}.
\end{itemize}

\subsection{Section 4: Managing Figures, Tables, and
Cross-References}\label{section-4-managing-figures-tables-and-cross-references}

While vanilla Markdown lacks a native method for sophisticated figure
management, this workflow integrates powerful tools for both creation
and referencing.

\textbf{Cross-Referencing with \texttt{pandoc-crossref}} This critical
academic function is seamlessly added by using the
\texttt{pandoc-crossref} filter
(\citeproc{ref-lierdakil2021crossref}{Lierdakil 2021}).

\textbf{Installation and Activation} First, the \texttt{pandoc-crossref}
executable must be installed. The filter is then activated by adding the
\texttt{-\/-filter\ pandoc-crossref} flag to the Pandoc command. If also
using the citation processor, this flag must appear \emph{before} the
\texttt{-\/-citeproc} flag.

\textbf{Labeling and Referencing Syntax} The syntax for labeling and
referencing elements is designed to be intuitive. For example, we can
reference the table of tools from earlier: Tab.~\ref{tbl:workbench}. We
can also reference the plot in Figure~\ref{fig:my-plot} and the famous
equation in eq.~\ref{eq:relativity}.

\begin{equation}\protect\phantomsection\label{eq:relativity}{E=mc^2}\end{equation}

\begin{figure}
\centering
\includegraphics[width=0.2\linewidth,height=\textheight,keepaspectratio,alt={Example plot}]{images/2025-11-17-19-47-43.png}
\caption{Example plot}\label{fig:my-plot}
\end{figure}

\textbf{Customization via YAML} The appearance of cross-references can
be customized through YAML metadata variables, as demonstrated in the
header of this file.

\textbf{Programmatic Diagramming with Mermaid} Beyond static images,
modern academic writing benefits from programmatic diagrams. This
workflow integrates \textbf{Mermaid.js}, allowing flowcharts and
diagrams to be defined directly in Markdown code blocks. The build
system includes a specialized pre-processing step that detects Mermaid
code blocks, renders them into high-resolution PNG images (scaled 3x for
crisp print quality) using a headless Chrome instance (Puppeteer) inside
the Docker container, and seamlessly substitutes them before the final
PDF generation. The flowchart in the Introduction of this paper is
generated using this exact method.

\section{Part III: Advanced Customization and Multilingual
Typesetting}\label{part-iii-advanced-customization-and-multilingual-typesetting}

\subsection{Section 5: Mastering Document Appearance with LaTeX
Templates}\label{section-5-mastering-document-appearance-with-latex-templates}

The visual appearance of the final PDF is controlled almost entirely by
LaTeX. Pandoc provides a powerful and flexible system for interfacing
with LaTeX templates.

\textbf{Using a Custom Template} For any serious academic work, a custom
template is usually required. A custom template is specified using the
\texttt{-\/-template} flag:
\texttt{pandoc\ mydoc.md\ -o\ mydoc.pdf\ -\/-template=eisvogel}.
High-quality templates can be found on publisher websites, in large
community repositories like Overleaf, and on code-hosting platforms like
GitHub (\citeproc{ref-wandmalfarbe2020eisvogel}{Wandmalfarbe 2020}).

\textbf{Case Study: The Eisvogel Template for a Custom Cover Page} The
popular Eisvogel template provides a clear example of how
template-specific variables, set in the YAML block, can be used for deep
customization (\citeproc{ref-wandmalfarbe2020eisvogel}{Wandmalfarbe
2020}).

\subsection{Section 6: A Guide to Traditional Chinese
Typesetting}\label{section-6-a-guide-to-traditional-chinese-typesetting}

Producing high-quality documents in Traditional Chinese requires
specific configuration. The Pandoc and LaTeX toolchain is exceptionally
capable in this regard.

\textbf{The Engine Requirement: \texttt{pdflatex} vs.~\texttt{xelatex}}
For any work involving non-Latin scripts, it is essential to use a
modern, Unicode-aware engine like \textbf{XeLaTeX}.

\textbf{Configuring Pandoc for \texttt{xelatex}} To instruct Pandoc to
use XeLaTeX, one can set \texttt{pdf-engine:\ xelatex} in the document's
YAML block.

\textbf{Font Selection for Traditional Chinese} The second critical
requirement is to specify a font that contains the necessary glyphs for
Traditional Chinese characters. This is also done via a variable in the
YAML block: \texttt{CJKmainfont:\ "Source\ Han\ Serif\ TC"}.

This configuration allows for seamless typesetting of Traditional
Chinese text, like this: 這是傳統中文的範例文字。

\subsection{Section 7: Granular Control over Page
Numbering}\label{section-7-granular-control-over-page-numbering}

Page numbering is a feature of the final typeset document, and as such,
its control resides entirely at the LaTeX level. Pandoc provides several
mechanisms to pass the necessary LaTeX commands from the Markdown source
to the final compilation stage. The cleanest method is using the
\texttt{header-includes} YAML field, as shown in this document's
metadata, to inject raw LaTeX commands like
\texttt{\textbackslash{}setcounter\{page\}\{1\}}.

\subsection{Section 8: Automated Multilingual Publishing with
LLMs}\label{section-8-automated-multilingual-publishing-with-llms}

A major barrier to internationalizing academic work is the effort
required to translate while preserving the rigorous formatting of the
manuscript. We have addressed this by integrating an \textbf{LLM-powered
translation pipeline}.

\textbf{The AI-Assisted Workflow}: By running \texttt{make\ zh\_tw}, the
system employs the Gemini API to translate both the Markdown manuscript
and the LaTeX cover page into Traditional Chinese. The pipeline is
designed to be \textbf{structure-aware}: it protects YAML metadata,
citation keys (\texttt{{[}@key{]}}), cross-reference labels
(\texttt{@fig:id}), and LaTeX commands, ensuring that the translated
output is immediately compilable.

\textbf{Challenges and Methodologies}: 1. \textbf{Preserving Structure}:
The system uses carefully crafted prompts to instruct the LLM to
translate only natural language text while leaving syntax markers
untouched. 2. \textbf{Font Handling}: The workflow automatically detects
and switches to CJK-compatible fonts (e.g., Noto Serif CJK TC) for the
translated build, avoiding ``tofu'' characters. 3.
\textbf{Cross-Reference Localization}: It translates the labels (e.g.,
``Figure'' to ``圖'') in the metadata, ensuring the scholarly apparatus
feels native to the target language.

This transforms the translation process from a manual typesetting
nightmare into a single-command automated task.

\section{Part IV: Historical Context and Modern
Practice}\label{part-iv-historical-context-and-modern-practice}

\subsection{Section 9: The `Old-Fashioned' Workflow: From Hot Metal to
Typewriters}\label{section-9-the-old-fashioned-workflow-from-hot-metal-to-typewriters}

Understanding the power of the modern plain-text workflow is enhanced by
appreciating the profound technical challenges it solves. Before the
advent of TeX and LaTeX, academic typesetting was a highly specialized,
manual process defined by severe physical and mechanical constraints.
The era of \textbf{hot metal typesetting} and later
\textbf{``typewriter'' composition} involved immense manual effort to
produce complex documents.

The development of TeX by Knuth in the late 1970s and LaTeX by Lamport
in the early 1980s marked a revolutionary shift
(\citeproc{ref-knuth1984tex}{Knuth 1984};
\citeproc{ref-lamport1986latex}{Lamport 1986}). They introduced the
concept of programmatic, algorithmic typesetting based on semantic
commands. The modern Pandoc user operates at an even higher level of
abstraction, using a simple, universal syntax that can be compiled to
LaTeX or other formats entirely.

\subsection{Section 10: Synthesis and Recommendations: Building Your
Sustainable
Workflow}\label{section-10-synthesis-and-recommendations-building-your-sustainable-workflow}

This report has detailed a comprehensive academic workflow. The very
file you are reading is a tangible example of this process. By combining
this Markdown file with the accompanying bibliography and
\texttt{Makefile}, you can reproduce the final PDF with a single
command. This demonstrates the power of a plain-text academic workflow:
unparalleled control, robust versioning, seamless collaboration, and the
assurance of future access to one's own work
(\citeproc{ref-rowleyWisdomHierarchyRepresentations2007}{Rowley 2007}).

\protect\phantomsection\label{refs}
\begin{CSLReferences}{1}{1}
\bibitem[\citeproctext]{ref-healy2018plain}
Healy, Kieran. 2018. {``The Plain Person's Guide to Plain Text Social
Science.''} In \emph{Unpublished Manuscript}. Preprint.
\url{https://kieranhealy.org/files/papers/plain-text.pdf}.

\bibitem[\citeproctext]{ref-knuth1984tex}
Knuth, Donald E. 1984. \emph{The TeXbook}. Reading, MA: Addison-Wesley.

\bibitem[\citeproctext]{ref-lamport1986latex}
Lamport, Leslie. 1986. \emph{LaTeX: A Document Preparation System}.
Reading, MA: Addison-Wesley.

\bibitem[\citeproctext]{ref-lierdakil2021crossref}
Lierdakil. 2021. {``Pandoc-Crossref: A Pandoc Filter for Numbering
Figures, Equations, Tables and Cross-References to Them.''}
\url{https://github.com/lierdakil/pandoc-crossref}.

\bibitem[\citeproctext]{ref-macfarlane2022pandoc}
MacFarlane, John. 2022. {``Pandoc - A Universal Document Converter.''}
\url{https://pandoc.org/}.

\bibitem[\citeproctext]{ref-rowleyWisdomHierarchyRepresentations2007}
Rowley, Jennifer. 2007. {``The Wisdom Hierarchy: Representations of the
DIKW Hierarchy.''} \emph{Journal of Information Science} 33 (2):
163--80. \url{https://doi.org/10.1177/0165551506070706}.

\bibitem[\citeproctext]{ref-wandmalfarbe2020eisvogel}
Wandmalfarbe, Eisvogel. 2020. {``Eisvogel: A Clean Pandoc LaTeX Template
to Convert Your Markdown Files to PDF or LaTeX.''}
\url{https://github.com/Wandmalfarbe/pandoc-latex-template}.

\end{CSLReferences}

\end{document}
