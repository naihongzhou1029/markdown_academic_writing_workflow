% Options for packages loaded elsewhere
\PassOptionsToPackage{unicode}{hyperref}
\PassOptionsToPackage{hyphens}{url}
\PassOptionsToPackage{space}{xeCJK}
%
\documentclass[
]{article}
\usepackage{amsmath,amssymb}
\usepackage{iftex}
\ifPDFTeX
  \usepackage[T1]{fontenc}
  \usepackage[utf8]{inputenc}
  \usepackage{textcomp} % provide euro and other symbols
\else % if luatex or xetex
  \usepackage{unicode-math} % this also loads fontspec
  \defaultfontfeatures{Scale=MatchLowercase}
  \defaultfontfeatures[\rmfamily]{Ligatures=TeX,Scale=1}
\fi
\usepackage{lmodern}
\ifPDFTeX\else
  % xetex/luatex font selection
  \ifXeTeX
    \usepackage{xeCJK}
    \setCJKmainfont[]{Noto Sans CJK SC}
          \fi
  \ifLuaTeX
    \usepackage[]{luatexja-fontspec}
    \setmainjfont[]{Noto Sans CJK SC}
  \fi
\fi
% Use upquote if available, for straight quotes in verbatim environments
\IfFileExists{upquote.sty}{\usepackage{upquote}}{}
\IfFileExists{microtype.sty}{% use microtype if available
  \usepackage[]{microtype}
  \UseMicrotypeSet[protrusion]{basicmath} % disable protrusion for tt fonts
}{}
\makeatletter
\@ifundefined{KOMAClassName}{% if non-KOMA class
  \IfFileExists{parskip.sty}{%
    \usepackage{parskip}
  }{% else
    \setlength{\parindent}{0pt}
    \setlength{\parskip}{6pt plus 2pt minus 1pt}}
}{% if KOMA class
  \KOMAoptions{parskip=half}}
\makeatother
\usepackage{xcolor}
\usepackage{longtable,booktabs,array}
\usepackage{calc} % for calculating minipage widths
% Correct order of tables after \paragraph or \subparagraph
\usepackage{etoolbox}
\makeatletter
\patchcmd\longtable{\par}{\if@noskipsec\mbox{}\fi\par}{}{}
\makeatother
% Allow footnotes in longtable head/foot
\IfFileExists{footnotehyper.sty}{\usepackage{footnotehyper}}{\usepackage{footnote}}
\makesavenoteenv{longtable}
\usepackage{graphicx}
\makeatletter
\def\maxwidth{\ifdim\Gin@nat@width>\linewidth\linewidth\else\Gin@nat@width\fi}
\def\maxheight{\ifdim\Gin@nat@height>\textheight\textheight\else\Gin@nat@height\fi}
\makeatother
% Scale images if necessary, so that they will not overflow the page
% margins by default, and it is still possible to overwrite the defaults
% using explicit options in \includegraphics[width, height, ...]{}
\setkeys{Gin}{width=\maxwidth,height=\maxheight,keepaspectratio}
% Set default figure placement to htbp
\makeatletter
\def\fps@figure{htbp}
\makeatother
\setlength{\emergencystretch}{3em} % prevent overfull lines
\providecommand{\tightlist}{%
  \setlength{\itemsep}{0pt}\setlength{\parskip}{0pt}}
\setcounter{secnumdepth}{5}
\newlength{\cslhangindent}
\setlength{\cslhangindent}{1.5em}
\newlength{\csllabelwidth}
\setlength{\csllabelwidth}{3em}
\newlength{\cslentryspacingunit} % times entry-spacing
\setlength{\cslentryspacingunit}{\parskip}
\newenvironment{CSLReferences}[2] % #1 hanging-ident, #2 entry spacing
 {% don't indent paragraphs
  \setlength{\parindent}{0pt}
  % turn on hanging indent if param 1 is 1
  \ifodd #1
  \let\oldpar\par
  \def\par{\hangindent=\cslhangindent\oldpar}
  \fi
  % set entry spacing
  \setlength{\parskip}{#2\cslentryspacingunit}
 }%
 {}
\usepackage{calc}
\newcommand{\CSLBlock}[1]{#1\hfill\break}
\newcommand{\CSLLeftMargin}[1]{\parbox[t]{\csllabelwidth}{#1}}
\newcommand{\CSLRightInline}[1]{\parbox[t]{\linewidth - \csllabelwidth}{#1}\break}
\newcommand{\CSLIndent}[1]{\hspace{\cslhangindent}#1}
\pagenumbering{arabic}
\setcounter{page}{1}
\usepackage{xeCJK}
\setCJKmainfont{Noto Sans CJK SC}
\usepackage[a4paper,margin=1in]{geometry}
\usepackage{etoolbox}
\AtBeginEnvironment{CSLReferences}{%
  \newpage\section*{References}%
  \setlength{\parindent}{0pt}%
}
\makeatletter
\@ifpackageloaded{subfig}{}{\usepackage{subfig}}
\@ifpackageloaded{caption}{}{\usepackage{caption}}
\captionsetup[subfloat]{margin=0.5em}
\AtBeginDocument{%
\renewcommand*\figurename{Figure}
\renewcommand*\tablename{Table}
}
\AtBeginDocument{%
\renewcommand*\listfigurename{List of Figures}
\renewcommand*\listtablename{List of Tables}
}
\newcounter{pandoccrossref@subfigures@footnote@counter}
\newenvironment{pandoccrossrefsubfigures}{%
\setcounter{pandoccrossref@subfigures@footnote@counter}{0}
\begin{figure}\centering%
\gdef\global@pandoccrossref@subfigures@footnotes{}%
\DeclareRobustCommand{\footnote}[1]{\footnotemark%
\stepcounter{pandoccrossref@subfigures@footnote@counter}%
\ifx\global@pandoccrossref@subfigures@footnotes\empty%
\gdef\global@pandoccrossref@subfigures@footnotes{{##1}}%
\else%
\g@addto@macro\global@pandoccrossref@subfigures@footnotes{, {##1}}%
\fi}}%
{\end{figure}%
\addtocounter{footnote}{-\value{pandoccrossref@subfigures@footnote@counter}}
\@for\f:=\global@pandoccrossref@subfigures@footnotes\do{\stepcounter{footnote}\footnotetext{\f}}%
\gdef\global@pandoccrossref@subfigures@footnotes{}}
\@ifpackageloaded{float}{}{\usepackage{float}}
\floatstyle{ruled}
\@ifundefined{c@chapter}{\newfloat{codelisting}{h}{lop}}{\newfloat{codelisting}{h}{lop}[chapter]}
\floatname{codelisting}{Listing}
\newcommand*\listoflistings{\listof{codelisting}{List of Listings}}
\makeatother
\ifLuaTeX
  \usepackage{selnolig}  % disable illegal ligatures
\fi
\IfFileExists{bookmark.sty}{\usepackage{bookmark}}{\usepackage{hyperref}}
\IfFileExists{xurl.sty}{\usepackage{xurl}}{} % add URL line breaks if available
\urlstyle{same}
\hypersetup{
  pdftitle={Sustainable Scholarship: A Robust Academic Workflow with Markdown, Pandoc, and LaTeX},
  pdfauthor={An Old-Fashioned Researcher},
  hidelinks,
  pdfcreator={LaTeX via pandoc}}

\title{Sustainable Scholarship: A Robust Academic Workflow with
Markdown, Pandoc, and LaTeX}
\author{An Old-Fashioned Researcher}
\date{2025-10-25}

\begin{document}
\maketitle
\begin{abstract}
This document serves as a complete, self-referential example of the
Pandoc academic workflow. It demonstrates the use of a YAML metadata
block for configuration, automated citations with pandoc-citeproc,
cross-references for figures, tables, and equations with
pandoc-crossref, and advanced customization for multilingual typesetting
and page numbering through LaTeX.
\end{abstract}

\hypertarget{introduction-the-philosophy-of-plain-text-academia}{%
\section{Introduction: The Philosophy of Plain-Text
Academia}\label{introduction-the-philosophy-of-plain-text-academia}}

The choice of writing tools in academic work is not merely a matter of
technical preference; it is a philosophical commitment to a particular
mode of scholarship. The modern plain-text workflow, centered on
Markdown, Pandoc, and LaTeX, represents a deliberate move towards a more
robust, transparent, and durable scholarly practice
(\protect\hyperlink{ref-healy2018plain}{Healy 2018}). This approach is
founded on a set of core principles that stand in stark contrast to the
opaque, proprietary nature of traditional word processors.

At its heart is the principle of \textbf{sustainability}. Plain-text
files, such as those written in Markdown, are the most resilient digital
format. Unlike the complex binary structures of \texttt{.docx} files,
which can become corrupted or obsolete as software changes, plain text
will remain readable and accessible for decades, ensuring the long-term
viability of one's intellectual output
(\protect\hyperlink{ref-healy2018plain}{Healy 2018};
\protect\hyperlink{ref-macfarlane2022pandoc}{MacFarlane 2022}). This is
coupled with the powerful concept of \textbf{separation of concerns},
where the semantic content of a document---the text, its structure, and
its meaning---is written in Markdown, completely divorced from its final
visual presentation, which is handled independently by LaTeX templates
or other styling mechanisms. This modularity allows for radical changes
in output format with zero alteration to the source manuscript.

Furthermore, this workflow is uniquely suited for the rigorous demands
of modern research. Plain-text files integrate seamlessly with
\textbf{version control systems} like Git, enabling meticulous tracking
of every change, non-destructive experimentation with drafts, and
transparent collaboration among authors---a process notoriously fraught
with difficulty when using binary files
(\protect\hyperlink{ref-healy2018plain}{Healy 2018}). The entire
process, from the initial draft to the final PDF, can be automated with
simple scripts, ensuring perfect \textbf{reproducibility} at any point
in the future, a cornerstone of scientific and scholarly integrity.

This system is best understood not as a collection of disparate tools,
but as a linear, modular data processing pipeline. The raw manuscript
(\texttt{.md}) and bibliographic data (\texttt{.json} or \texttt{.bib})
serve as the initial inputs. These inputs are then passed through a
chain of specialized filters and transformers: Zotero and its Better
BibTeX extension manage and export bibliographic data; Pandoc parses the
source text (\protect\hyperlink{ref-macfarlane2022pandoc}{MacFarlane
2022}); \texttt{pandoc-citeproc} resolves citation markers;
\texttt{pandoc-crossref} numbers figures and equations
(\protect\hyperlink{ref-lierdakil2021crossref}{Lierdakil 2021}); and
finally, a LaTeX engine like XeLaTeX performs the final typesetting to
produce a PDF. Each stage is discrete and transparent. This contrasts
fundamentally with the monolithic, ``black box'' environment of a word
processor, where these processes are intertwined and hidden from the
user. The power of this workflow lies in the ability to control this
pipeline, to swap out components, insert new processing stages, and
debug any issue by inspecting the intermediate output at any point---for
instance, by generating the intermediate \texttt{.tex} file to diagnose
a LaTeX error. This level of control and transparency is the key to
solving the complex, bespoke formatting challenges inherent in academic
writing.

\hypertarget{part-i-the-core-toolchain}{%
\section{Part I: The Core Toolchain}\label{part-i-the-core-toolchain}}

\hypertarget{section-1-assembling-your-digital-workbench}{%
\subsection{Section 1: Assembling Your Digital
Workbench}\label{section-1-assembling-your-digital-workbench}}

Before embarking on the plain-text writing process, a one-time setup of
the core toolchain is required. This digital workbench forms the
foundation of the entire workflow.

\textbf{Pandoc: The Universal Document Converter} At the heart of the
workflow is Pandoc, a command-line utility often described as the
``swiss army knife'' for document conversion
(\protect\hyperlink{ref-macfarlane2022pandoc}{MacFarlane 2022}). It is
responsible for parsing the source Markdown file and orchestrating its
transformation into the final output format.

\textbf{A LaTeX Distribution: The Typesetting Engine} Pandoc does not
create PDF files directly. Instead, it generates LaTeX source code
(\texttt{.tex}), which is then compiled by a dedicated LaTeX engine to
produce the final, professionally typeset PDF. To ensure all necessary
packages for advanced features, complex layouts, and multilingual
support are available, it is strongly recommended to install a full
LaTeX distribution, such as TeX Live.

\textbf{A Plain-Text Editor: The Writing Environment} The writing itself
is done in a plain-text editor. Modern, extensible editors like Visual
Studio Code (VSCode), Atom, or the academic-focused Zettlr are ideal
choices.

\textbf{Zotero: The Reference Manager} Robust reference management is
handled by Zotero, a powerful, open-source application. For this
workflow, the installation of the \textbf{Better BibTeX for Zotero
(BBT)} extension is non-negotiable. BBT provides two essential features:
the automatic generation of stable, human-readable citation keys and a
highly reliable, automated export process that keeps the bibliography
file synchronized with the Zotero library.

The components are summarized in Tab.~\ref{tbl:workbench}.

\begin{longtable}[]{@{}
  >{\raggedright\arraybackslash}p{(\columnwidth - 6\tabcolsep) * \real{0.1006}}
  >{\raggedright\arraybackslash}p{(\columnwidth - 6\tabcolsep) * \real{0.3128}}
  >{\raggedright\arraybackslash}p{(\columnwidth - 6\tabcolsep) * \real{0.1117}}
  >{\raggedright\arraybackslash}p{(\columnwidth - 6\tabcolsep) * \real{0.4749}}@{}}
\caption{\label{tbl:workbench}The Digital Scholar's
Workbench.}\tabularnewline
\toprule\noalign{}
\begin{minipage}[b]{\linewidth}\raggedright
Component
\end{minipage} & \begin{minipage}[b]{\linewidth}\raggedright
Purpose
\end{minipage} & \begin{minipage}[b]{\linewidth}\raggedright
Recommended Software
\end{minipage} & \begin{minipage}[b]{\linewidth}\raggedright
Key Configuration Notes
\end{minipage} \\
\midrule\noalign{}
\endfirsthead
\toprule\noalign{}
\begin{minipage}[b]{\linewidth}\raggedright
Component
\end{minipage} & \begin{minipage}[b]{\linewidth}\raggedright
Purpose
\end{minipage} & \begin{minipage}[b]{\linewidth}\raggedright
Recommended Software
\end{minipage} & \begin{minipage}[b]{\linewidth}\raggedright
Key Configuration Notes
\end{minipage} \\
\midrule\noalign{}
\endhead
\bottomrule\noalign{}
\endlastfoot
Document Converter & Parses Markdown and orchestrates the conversion
process. & Pandoc & Install via OS-specific package manager (e.g.,
Homebrew for macOS). \\
Typesetting Engine & Compiles LaTeX code generated by Pandoc into a PDF.
& TeX Live & Install the full distribution to avoid missing package
errors. \\
Text Editor & The environment for writing in plain-text Markdown. &
VSCode & Install extensions: Markdown Preview Enhanced and Pandoc
Citer. \\
Reference Manager & Manages bibliographic data and exports it for
Pandoc. & Zotero & Install the Better BibTeX for Zotero (BBT) extension
for stable keys and auto-export. \\
\end{longtable}

\hypertarget{section-2-the-pandoc-conversion-engine-from-markdown-to-pdf}{%
\subsection{Section 2: The Pandoc Conversion Engine: From Markdown to
PDF}\label{section-2-the-pandoc-conversion-engine-from-markdown-to-pdf}}

With the toolchain installed, the conversion process is driven by the
Pandoc command-line interface, configured primarily through a metadata
block within the Markdown file itself, as seen at the top of this very
document.

\textbf{The Basic Conversion Command} The fundamental command to convert
a Markdown file to a PDF is straightforward:
\texttt{pandoc\ input.md\ -o\ output.pdf}. Pandoc typically infers the
input and output formats from the file extensions.

\textbf{The YAML Metadata Block: The Document's Control Panel} Rather
than relying on long and cumbersome command-line flags, Pandoc
configurations are best managed within a YAML metadata block at the very
top of the Markdown file, delimited by \texttt{-\/-\/-} on either side.
This approach is superior because it keeps the document's essential
metadata and its compilation settings version-controlled alongside the
content itself.

\textbf{The Standalone Flag (\texttt{-s})} A critical option for
generating a complete document is \texttt{-\/-standalone} (or its
shorthand, \texttt{-s}). This flag instructs Pandoc to use a template to
wrap the converted content with the necessary header and footer
material---for example, the
\texttt{\textbackslash{}documentclass\{...\}} and
\texttt{\textbackslash{}begin\{document\}...\textbackslash{}end\{document\}}
commands in LaTeX---to create a self-contained, compilable file rather
than a mere fragment.

\hypertarget{part-ii-managing-scholarly-apparatus}{%
\section{Part II: Managing Scholarly
Apparatus}\label{part-ii-managing-scholarly-apparatus}}

\hypertarget{section-3-automated-citations-and-bibliographies-with-pandoc-citeproc}{%
\subsection{\texorpdfstring{Section 3: Automated Citations and
Bibliographies with
\texttt{pandoc-citeproc}}{Section 3: Automated Citations and Bibliographies with pandoc-citeproc}}\label{section-3-automated-citations-and-bibliographies-with-pandoc-citeproc}}

A cornerstone of academic writing is the correct management of citations
and bibliographies. The Pandoc workflow automates this process with
exceptional precision using its citation processor,
\texttt{pandoc-citeproc}.

\textbf{The Role of \texttt{pandoc-citeproc}} The citation processor,
invoked with the \texttt{-\/-citeproc} flag in modern Pandoc versions,
is the filter responsible for parsing Markdown citation syntax,
transforming it into fully formatted in-text citations, and
automatically generating a bibliography at the end of the document.

\textbf{Step 1: Configuring Zotero and Better BibTeX (BBT)} The process
begins with the reference manager. Within Zotero, the Better BibTeX
extension should be configured to automatically export the desired
library or collection to a file whenever an entry is added or modified.

\textbf{Step 2: Choosing the Bibliography Format: CSL JSON vs.~BibTeX}
To ensure the highest possible fidelity between the reference manager
and the final document, it is strongly recommended to export directly
from Zotero to a CSL-native format, such as \textbf{Better CSL JSON} or
\textbf{Better CSL YAML}. This eliminates the ``man-in-the-middle''
conversion and preserves the integrity of the bibliographic data.

\textbf{Step 3: Specifying Bibliography and Style in YAML} The
connection between the Markdown document and the bibliographic data is
made in the YAML metadata block. Two keys are required:
\texttt{bibliography:} and \texttt{csl:}. Thousands of CSL styles for
various journals and conventions are available for download from the
official Zotero Style Repository.

\textbf{Step 4: Mastering Pandoc Citation Syntax} With the setup
complete, citing sources in Markdown is simple and expressive. The
syntax supports a wide range of scholarly conventions.

\begin{itemize}
\item
  \textbf{Standard Parenthetical Citations:}
  \texttt{...as\ has\ been\ shown\ {[}@knuth1984tex;\ @lamport1986latex{]}}.
\item
  \textbf{Narrative (Author-in-Text) Citations:}
  \texttt{@macfarlane2022pandoc\ argues\ that...} renders as
  ``MacFarlane (2022) argues that\ldots{}''.
\item
  \textbf{Suppressing the Author:} When an author is already mentioned,
  their name can be suppressed in the citation by adding a minus sign:
  \texttt{Healy\textquotesingle{}s\ research\ {[}-@healy2018plain{]}\ confirms...}
  renders as ``Healy's research (2018) confirms\ldots{}''.
\item
  \textbf{Prefixes, Locators, and Suffixes:}
  \texttt{{[}see\ @knuth1984tex,\ chap.\ 3{]}}.
\end{itemize}

\hypertarget{section-4-cross-referencing-figures-tables-and-equations-with-pandoc-crossref}{%
\subsection{\texorpdfstring{Section 4: Cross-Referencing Figures,
Tables, and Equations with
\texttt{pandoc-crossref}}{Section 4: Cross-Referencing Figures, Tables, and Equations with pandoc-crossref}}\label{section-4-cross-referencing-figures-tables-and-equations-with-pandoc-crossref}}

While vanilla Markdown lacks a native method for numbering and
cross-referencing, this critical academic function is seamlessly added
by using the \texttt{pandoc-crossref} filter
(\protect\hyperlink{ref-lierdakil2021crossref}{Lierdakil 2021}).

\textbf{Installation and Activation} First, the \texttt{pandoc-crossref}
executable must be installed. The filter is then activated by adding the
\texttt{-\/-filter\ pandoc-crossref} flag to the Pandoc command. If also
using the citation processor, this flag must appear \emph{before} the
\texttt{-\/-citeproc} flag.

\textbf{Labeling and Referencing Syntax} The syntax for labeling and
referencing elements is designed to be intuitive. For example, we can
reference the table of tools from earlier: Tab.~\ref{tbl:workbench}. We
can also reference the plot in Figure~\ref{fig:my-plot} and the famous
equation in eq.~\ref{eq:relativity}.

\begin{equation}\protect\hypertarget{eq:relativity}{}{E=mc^2}\end{equation}

\begin{figure}
\hypertarget{fig:my-plot}{%
\centering
\includegraphics[width=0.2\textwidth,height=\textheight]{images/2025-11-17-19-47-43.png}
\caption{Example plot}\label{fig:my-plot}
}
\end{figure}

\textbf{Customization via YAML} The appearance of cross-references can
be customized through YAML metadata variables, as demonstrated in the
header of this file.

\hypertarget{part-iii-advanced-customization-and-multilingual-typesetting}{%
\section{Part III: Advanced Customization and Multilingual
Typesetting}\label{part-iii-advanced-customization-and-multilingual-typesetting}}

\hypertarget{section-5-mastering-document-appearance-with-latex-templates}{%
\subsection{Section 5: Mastering Document Appearance with LaTeX
Templates}\label{section-5-mastering-document-appearance-with-latex-templates}}

The visual appearance of the final PDF is controlled almost entirely by
LaTeX. Pandoc provides a powerful and flexible system for interfacing
with LaTeX templates.

\textbf{Using a Custom Template} For any serious academic work, a custom
template is usually required. A custom template is specified using the
\texttt{-\/-template} flag:
\texttt{pandoc\ mydoc.md\ -o\ mydoc.pdf\ -\/-template=eisvogel}.
High-quality templates can be found on publisher websites, in large
community repositories like Overleaf, and on code-hosting platforms like
GitHub (\protect\hyperlink{ref-wandmalfarbe2020eisvogel}{Wandmalfarbe
2020}).

\textbf{Case Study: The Eisvogel Template for a Custom Cover Page} The
popular Eisvogel template provides a clear example of how
template-specific variables, set in the YAML block, can be used for deep
customization
(\protect\hyperlink{ref-wandmalfarbe2020eisvogel}{Wandmalfarbe 2020}).

\hypertarget{section-6-a-guide-to-traditional-chinese-typesetting}{%
\subsection{Section 6: A Guide to Traditional Chinese
Typesetting}\label{section-6-a-guide-to-traditional-chinese-typesetting}}

Producing high-quality documents in Traditional Chinese requires
specific configuration. The Pandoc and LaTeX toolchain is exceptionally
capable in this regard.

\textbf{The Engine Requirement: \texttt{pdflatex} vs.~\texttt{xelatex}}
For any work involving non-Latin scripts, it is essential to use a
modern, Unicode-aware engine like \textbf{XeLaTeX}.

\textbf{Configuring Pandoc for \texttt{xelatex}} To instruct Pandoc to
use XeLaTeX, one can set \texttt{pdf-engine:\ xelatex} in the document's
YAML block.

\textbf{Font Selection for Traditional Chinese} The second critical
requirement is to specify a font that contains the necessary glyphs for
Traditional Chinese characters. This is also done via a variable in the
YAML block: \texttt{CJKmainfont:\ "Source\ Han\ Serif\ TC"}.

This configuration allows for seamless typesetting of Traditional
Chinese text, like this: 這是傳統中文的範例文字。

\hypertarget{section-7-granular-control-over-page-numbering}{%
\subsection{Section 7: Granular Control over Page
Numbering}\label{section-7-granular-control-over-page-numbering}}

Page numbering is a feature of the final typeset document, and as such,
its control resides entirely at the LaTeX level. Pandoc provides several
mechanisms to pass the necessary LaTeX commands from the Markdown source
to the final compilation stage. The cleanest method is using the
\texttt{header-includes} YAML field, as shown in this document's
metadata, to inject raw LaTeX commands like
\texttt{\textbackslash{}setcounter\{page\}\{1\}}.

\hypertarget{part-iv-historical-context-and-modern-practice}{%
\section{Part IV: Historical Context and Modern
Practice}\label{part-iv-historical-context-and-modern-practice}}

\hypertarget{section-8-the-old-fashioned-workflow-from-hot-metal-to-typewriters}{%
\subsection{Section 8: The `Old-Fashioned' Workflow: From Hot Metal to
Typewriters}\label{section-8-the-old-fashioned-workflow-from-hot-metal-to-typewriters}}

Understanding the power of the modern plain-text workflow is enhanced by
appreciating the profound technical challenges it solves. Before the
advent of TeX and LaTeX, academic typesetting was a highly specialized,
manual process defined by severe physical and mechanical constraints.
The era of \textbf{hot metal typesetting} and later
\textbf{``typewriter'' composition} involved immense manual effort to
produce complex documents.

The development of TeX by Knuth in the late 1970s and LaTeX by Lamport
in the early 1980s marked a revolutionary shift
(\protect\hyperlink{ref-knuth1984tex}{Knuth 1984};
\protect\hyperlink{ref-lamport1986latex}{Lamport 1986}). They introduced
the concept of programmatic, algorithmic typesetting based on semantic
commands. The modern Pandoc user operates at an even higher level of
abstraction, using a simple, universal syntax that can be compiled to
LaTeX or other formats entirely.

\hypertarget{section-9-synthesis-and-recommendations-building-your-sustainable-workflow}{%
\subsection{Section 9: Synthesis and Recommendations: Building Your
Sustainable
Workflow}\label{section-9-synthesis-and-recommendations-building-your-sustainable-workflow}}

This report has detailed a comprehensive academic workflow. The very
file you are reading is a tangible example of this process. By combining
this Markdown file with the accompanying bibliography and
\texttt{Makefile}, you can reproduce the final PDF with a single
command. This demonstrates the power of a plain-text academic workflow:
unparalleled control, robust versioning, seamless collaboration, and the
assurance of future access to one's own work
(\protect\hyperlink{ref-rowleyWisdomHierarchyRepresentations2007}{Rowley
2007}).

\hypertarget{refs}{}
\begin{CSLReferences}{1}{0}
\leavevmode\vadjust pre{\hypertarget{ref-healy2018plain}{}}%
Healy, Kieran. 2018. {``The Plain Person's Guide to Plain Text Social
Science.''} In \emph{Unpublished Manuscript}. Preprint.
\url{https://kieranhealy.org/files/papers/plain-text.pdf}.

\leavevmode\vadjust pre{\hypertarget{ref-knuth1984tex}{}}%
Knuth, Donald E. 1984. \emph{The TeXbook}. Reading, MA: Addison-Wesley.

\leavevmode\vadjust pre{\hypertarget{ref-lamport1986latex}{}}%
Lamport, Leslie. 1986. \emph{LaTeX: A Document Preparation System}.
Reading, MA: Addison-Wesley.

\leavevmode\vadjust pre{\hypertarget{ref-lierdakil2021crossref}{}}%
Lierdakil. 2021. {``Pandoc-Crossref: A Pandoc Filter for Numbering
Figures, Equations, Tables and Cross-References to Them.''}
\url{https://github.com/lierdakil/pandoc-crossref}.

\leavevmode\vadjust pre{\hypertarget{ref-macfarlane2022pandoc}{}}%
MacFarlane, John. 2022. {``Pandoc - A Universal Document Converter.''}
\url{https://pandoc.org/}.

\leavevmode\vadjust pre{\hypertarget{ref-rowleyWisdomHierarchyRepresentations2007}{}}%
Rowley, Jennifer. 2007. {``The Wisdom Hierarchy: Representations of the
DIKW Hierarchy.''} \emph{Journal of Information Science} 33 (2):
163--80. \url{https://doi.org/10.1177/0165551506070706}.

\leavevmode\vadjust pre{\hypertarget{ref-wandmalfarbe2020eisvogel}{}}%
Wandmalfarbe, Eisvogel. 2020. {``Eisvogel: A Clean Pandoc LaTeX Template
to Convert Your Markdown Files to PDF or LaTeX.''}
\url{https://github.com/Wandmalfarbe/pandoc-latex-template}.

\end{CSLReferences}

\end{document}
